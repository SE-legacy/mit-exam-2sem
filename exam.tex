\documentclass[onecolumn]{article}

\usepackage {preamble}

\begin{document}

\section{\LaTeX}

\subsection{Понятие класса документа. Примеры классов документов}
На тип документа указывает команда \verb|\documentclass{}|.
Основные типы документов:
\begin{itemize}
\item article "--- для оформления статей;
\item book "--- для оформления книг;
\item letter "--- для оформления писем;
\item beamer "--- для оформления презентаций;
\item CSWorks "--- для оформления научных работ на факультете КНиИТ;
\end{itemize}

У команды \verb|\documentclass| есть параметры, которые указываются в квадратных скобках до фигурных скобок. Основные параметры:
\begin{itemize}
\item размер шрифта "--- по умолчанию 10pt;
\item тип бумаги (a4paper, letterpaper) "--- по умолчанию letterpaper;
\item нужна ли пустая страница после титульного листа (titlepage, notitlepage) "--- для article по умолчанию notitlepage;
\item тип печати (oneside, twoside);
\item количество колонок (onecolumn, twocolumn);
\end{itemize}

\subsection{Понятие преамбулы документа, подключение пакетов. Основные пакеты для базовой компиляции документа на русском языке}
Сам документ заключён между командами \verb|\begin{document}| и \verb|\end{document}|, которые называются окружением. Перед началом окружения располагается преамбула, которая задаёт правила форматирования и другие настройки. Пакеты подключаются с помощью команды \verb|\usepackage|. Можно вынести преамбулу в отдельный файл \textit{preamble.sty} и подключить его из основного файла командой \verb|\usepackage{preamble}|. Для отображения русского языка необходимо подключить следующие три пакета:\\
\verb|\usepackage[T2A] {fontenc}|\\
\verb|\usepackage[utf8] {inputenc}|\\
\verb|\usepackage[russian] {babel}|\\

\subsection{Секционирование документа (основные команды). Взаимодействие секционирования с результатом выполнения команды tableofcontents. Структурирование документа (разделение на поддокументы с последующей компиляцией воедино). Хорошие практики при структурировании документа}
Чтобы соблюдать логические границы между различными частями документа, существует секционирование, то есть разделение документа на секции. Имеет место вложенность глава (только для книги) $\rightarrow$ секция $\rightarrow$ подсекция $\rightarrow$ подподсекция. Используются следующие обозначения:
\begin{itemize}
\item \verb|\chapter{}| "--- глава, где в фигурных скобках при необходимости пишется название;
\item \verb|\chapter*{}| "--- глава, которой не присваивается номер (без названия не имеет смысла);
\item \verb|\section(*){}| "--- секция, которой присваивается (соотв., не присваивается) номер;
\item \verb|\subsection(*){}| "--- подсекция, которой присваивается (соотв., не присваивается) номер;
\item \verb|\subsubsection(*){}| "--- подподсекция, которой присваивается (соотв., не присваивается) номер;
\end{itemize}

Создание новой секции в статье или главы в книге автоматически заканчивает текущую и начинает новую страницу.

В документе можно создать содержание, куда автоматически включаются нумерованные главы, секции и подсекции, с помощью команды \verb|\tableofcontents|.

Хорошей практикой является разделение документа на поддокументы. Для этого части документа сохраняются в отдельные файлы, а затем подключаются с помощью команды \verb|input{}|, где в фигурных скобках указывается название файла и путь к нему, если он лежит не в текущей директории. Если подключаемый файл имеет формат \textit{.tex}, то содержимое этого файла будет предварительно скомпилировано.

\subsection{Форматирование текста. Базовые команды для оформления знаков препинания: тире, дефиса, кавычек. Команды для оформления жирного и курсивного шрифтов}
Абзацы отделяются друг от друга пустой строкой, перенос на новую строку осуществляют команды \verb|\newline| и \verb|\\|, перенос на новую страницу "--- \verb|\newpage|. 

После разделительных знаков препинания, но не перед ними, ставится пробел; скобки с обеих сторон выделяются пробелами, вокруг дефиса пробелы не ставятся. Некоторые знаки препинания записываются особым образом:
\begin{itemize}
\item тире: \verb|"---| или \verb|~---|;
\item числовые интервалы: \verb|--|;
\item дефис: \verb|"=|;
\item кавычки-ёлочки: \verb|<<text>>|;
\end{itemize}

Существуют команды, изменяющие размер шрифта:
\begin{itemize}
\item \verb|\tiny| (5pt);
\item \verb|\scriptsize| (7pt);
\item \verb|\footnotesize| (8pt);
\item \verb|\small| (9pt);
\item \verb|\normalsize| (10pt);
\item \verb|\large| (12pt);
\item \verb|\Large| (14pt);
\item \verb|\LARGE| (17pt);
\item \verb|\huge| (20 pt);
\item \verb|\Huge| (25pt);
\end{itemize}

\textbf{Жирный} шрифт оформляется с помощью команды \verb|\textbf|, а \textit{курсив} "--- с помощью команды \verb|\textit|. Команда \verb~\verb|command|~ позволяет вставить в документ команду \LaTeX без её выполнения. Окружения flushleft, center и flushright обеспечивают выравнивание по левому краю, по центру и по правому краю соответственно.

\subsection{Понятие списка. Виды списков, оформление вложенных списков}
Список "--- перечисление элементов. Если элементы списка представляют собой целые предложения, то в конце каждого элемента обычно ставится точка, если же это слова или словосочетания, они разделяются точками с запятой. Элементы списка разделяются с помощью команды \verb|\item|. 
	
Существует три вида окружения для списков:
\begin{itemize}
\item \verb|\begin{itemize} \end{itemize}| "--- создаёт ненумерованный список;
\item \verb|\begin{enumerate} \end{enumerate}| "--- создаёт нумерованный список;
\item \verb|\begin{description} \end{description}| "--- создаёт список"=описание (в таком случае каждый элемент начинается как \verb|\item[заголовок элемента]|.
\end{itemize}

\subsection{Понятие математического окружения. Синтаксис. Встроенные формулы и выносные формулы. Особенности математического окружения: работа с отступами, кириллицей}
Для того, чтобы оформлять математические формулы, в \LaTeX есть математическое окружение. Существует два типа формул: встроенные и выносные.

Встроенные формулы располагаются внутри текста, они выглядят как команды, заключённые в два знака \$ (\verb|$equation$|).

Для выносных формул можно использовать окружение \verb|\begin{equation} \end{equation}| (для нумерованных формул) или \verb|\begin{equation*} \end{equation*}| (для ненумерованных формул). Окружение \verb|\begin{align} \end{align}| позволяет выравнивать длинные формулы.

Верхний индекс обозначается с помощью символа \^, а нижний "--- с помощью символа \_. Индексы из нескольких символов заключаются в фигурные скобки.

Знак умножения ставится с помощью команды \verb|$\cdot$|, дробь записывается как \verb|$\frac{}{}$|. Знаки сравнения записываются как \verb|$\le$| ($\le$), \verb|$\leqslant$| ($\leqslant$), \verb|$\ge$| ($\ge$), verb|$\geqslant$| ($\geqslant$), \verb|$\ne$| ($\ne$).

Некоторые примеры команд математического окружения:
\begin{itemize}
\item Греческие буквы: \verb|$\alpha$| ($\alpha$), \verb|$epsilon$| ($\epsilon$), \verb|$varepsilon$| ($\varepsilon$);
\item Сумма, произведение, интеграл: \verb|$\sum$| ($\sum$), \verb|$prod$| ($\prod$), \verb|$\int$| ($\int$);
\item Корень n"=ной степени: \verb|$\sqrt[n]{}$| ($\sqrt[3]{x}$);
\end{itemize}

Чтобы в математическом режиме отображался текст, его нужно оборачивать в команду \verb|$\text{}$|, для отображения пробелов применяется тильда \~.

\subsection{Работа со сложными объектами. Базовые команды для создания матриц, таблиц, систем уравнений. Синтаксис}

Для создания матриц используются следующие окружения:
\begin{itemize}
\item \verb|\begin{matrix} \end{matrix}| "--- матрица без скобок;
\item \verb|\begin{bmatrix} \end{bmatrix}| "--- матрица в квадратных скобках [ ];
\item \verb|\begin{pmatrix} \end{pmatrix}| "--- матрица в круглых скобках ( );
\item \verb|\begin{vmatrix} \end{vmatrix}| "--- матрица в прямых скобках | |;
\item \verb|\begin{Bmatrix} \end{Bmatrix}| "--- матрица в фигурных скобках { };
\item \verb|\begin{Vmatrix} \end{Vmatrix}| "--- матрица в двойных прямых скобках || ||;
\end{itemize}

Элементы матрицы в рамках одной строки разделяются символом \&, а строки матрицы "--- символом \verb|\\|.

Простая таблица создаётся с помощью окружения \verb|\begin{tabular} \end{tabular}|, если нужна подпись и ссылка, используется окружение \verb|\begin{table} \end{table}|. Разделение элементов выполняется так же, как в матрице.

Для создания систем уравнений применяется окружение \verb|\begin{cases} \end{cases}|.

\section{Работа с программным кодом. Аргументы команд пакета minted или Verbatim (fancyvrb). Основные опциональные параметры}
Для вставки кода можно использовать пакет minted, который подключается с помощью команды \verb|\usepackage{minted}|. Когда код пишется в теле документа, он выделяется с помощью команды \verb|\begin{minted}[options]{language} \end{minted}|, а для подключения кода из отдельного файла используется синтаксис \verb|\inputminted[options]{language}{filename}|.
Основные опциональные параметры:
\begin{itemize}
\item frame=lines/leftline/topline/bottomline "--- рисует линии вокруг кода, чтобы выделить его;
\item framesep=2mm/pt "--- дистанция между текстом и окружением кода;
\item baselinestretch "--- дистанция между строками;
\item bgcolor "--- цвет фона;
\item fontsize "--- размер шрифта;
\item style=bw "--- установка чёрно-белого стиля кода;
\item linenos "--- включение отображения номеров строк;
\end{itemize}

\section{Вставка изображений. Окружение figure, позиционирование плавающих объектов. Масштабирование изображений}
Плавающие объекты вставляются с помощью окружения \verb|\begin{figure} \end{figure}|. Внутри этого окружения существует команда для центрирования объекта \verb|\centering|, а также параметры, отвечающие за то, как объект будет вставлен в документ:
\begin{itemize}
\item h "--- вставка на месте, т.е. ближе всего к реальному расположению объекта в документе;
\item H "--- вставка \textbf{ровно} на том месте, где объект указан в документе;
\item t (b) "--- вставка вверху (внизу) страницы;
\item p "--- вставка на специальной странице для плавающих объектов;
\item ! "--- попытаться перегрузить внутренние параметры \LaTeX, чтобы определить <<хорошую>> позицию;
\end{itemize}

Рисунки вставляются командой \verb|\includegraphics[options]{file}|. Для изменения относительного размера изображения используется параметр scale. Если он равен $1.0$, изображение не изменит масштаб, если он равен $2.0$ "--- увеличится в 2 раза, $0.5$ "--- уменьшится в 2 раза. Команда \verb|\caption| вставляет подпись к рисунку.

\end{document}
